% !TEX TS-program = xelatex
% !TEX encoding = UTF-8

% \documentclass[a4paper,10pt,oneside]{article}
\documentclass[letterpaper,10pt,oneside]{simpleresumecv}

%%%%%%%%%%%%%%%%%%%%%%%%%%%%%%%%%%%%%%%%%%%%%%%%%%%%%%%%%%%%%%%%%
%% PREAMBLE.
%%%%%%%%%%%%%%%%%%%%%%%%%%%%%%%%%%%%%%%%%%%%%%%%%%%%%%%%%%%%%%%%%

% CV Info (to be customized).
\newcommand{\CVAuthor}{Devin Lehmacher}
\newcommand{\CVTitle}{Devin Lehmacher's Resume for Apple}
\newcommand{\CVNote}{Resume compiled on {\today} for Apple}
\newcommand{\CVWebpage}{github.com/lehmacdj}

% PDF settings and properties.
\hypersetup{%
pdftitle={\CVTitle},
pdfauthor={\CVAuthor},
pdfsubject={\CVWebpage},
pdfcreator={XeLaTeX},
pdfproducer={},
pdfkeywords={},
pdfpagemode={},
bookmarks=true,
unicode=true,
bookmarksopen=true,
pdfstartview=FitH,
pdfpagelayout=OneColumn,
pdfpagemode=UseOutlines,
hidelinks,
breaklinks}

% Shorthand.
\newcommand{\CodeCommand}[1]{\mbox{\textbf{\textbackslash{#1}}}}

%\newcommand{\CS3110}

%%%%%%%%%%%%%%%%%%%%%%%%%%%%%%%%%%%%%%%%%%%%%%%%%%%%%%%%%%%%%%%%%
%% ACTUAL DOCUMENT.
%%%%%%%%%%%%%%%%%%%%%%%%%%%%%%%%%%%%%%%%%%%%%%%%%%%%%%%%%%%%%%%%%

\begin{document}

%%%%%%%%%%%%%%%
% TITLE BLOCK %
%%%%%%%%%%%%%%%

\title{\CVAuthor}

\begin{subtitle}
\href{https://www.google.com/maps/place/232+Kings+Way,+Clemson,+SC+29631,+USA}
{232 Kings Way, Clemson, SC 29631, USA}
\par
\href{mailto:djl329@cornell.edu}
{djl329@cornell.edu}
\,\SubBulletSymbol\,
+1\,(864)\,722--3014
\,\SubBulletSymbol\,
\href{https://\CVWebpage}
{\CVWebpage}
\end{subtitle}

\begin{body}

%%%%%%%%%%%%%%%
%% EDUCATION %%
%%%%%%%%%%%%%%%

\section%
{Education}
{Education}
{PDF:Education}

\href{https://www.cornell.edu}
{\textbf{Cornell University}},
Ithaca, NY 14853
\hfill
\DatestampYMD{2015}{08}{25} --- Present
\BulletItem%
Bachelors of Arts, degree anticipated
\DatestampYM{2019}{05}
\begin{detail}
\SubBulletItem%
Majors:
\href{https://www.cs.cornell.edu}
{Computer Science}
and
\href{https://www.biology.cornell.edu}
{Biology}
\SubBulletItem%
Cumulative GPA:\@ 3.39
\end{detail}

%%%%%%%%%%%%%%%%%%%%%%%%%
%% RESEARCH EXPERIENCE %%
%%%%%%%%%%%%%%%%%%%%%%%%%

\section%
{Classes}
{Classes}
{PDF:Classes}

\href{https://www.cs.cornell.edu/courses/cs3110/2016sp/}
{{\textbf{CS 3110}}: Functional Programming and Data Structures}
\hfill Spring 2016
\BulletItem%
OCaml: Standard library, Async, Functional programming techniques
% I have since learned OCaml Core this should be in the sideproject section
% where I have a part about OCaml Ed
% Using options and monads to make life easier
% Statelessness allows for much better concurrency and makes programs easier to
% reason about
\BulletItem%
Type theory, propositional logic, constructive real numbers, convex hull problem
% Value assurance, error handling using option types, monads, functors,
% assurance of invariants through the use of variant checking functions
% Proofs of propositional logic statements using evidence of the existence of a
% function
% Bishop book,
\BulletItem%
Data Structures: Splay trees, Monads, Modules, Functors
% Self balancing BST where most recently used stuff is near to top
% join, bind, return: important for usage of async
% modules as abstraction
% functors as ways to implement generic stuff and conform to protocols
\BulletItem%
Language evaluation: wrote an interpreter for a subset of OCaml
% Small step evaluation: follow order of operations from the leaves of the AST
% Big step evaluation: environment and such stuff; we used this to implement
% OCalf
\GapNoBreak%

\href{https://www.cs.cornell.edu/courses/cs3410/2016fa/}
{\textbf{CS 3410}}: Computer System Organization and Programming
\hfill Fall 2016
\BulletItem%
Digital design using Logisim, MIPS assembly, C
\BulletItem%
Processor design: built a fully pipelined MIPS processor
instructions
\BulletItem%
Memory management: implementation of malloc in C
\GapNoBreak%

\href{https://www.cs.cornell.edu/courses/cs2110/2015fa/}
{\textbf{CS 2110}}: Object Oriented Programming and Data Structures
\hfill Fall 2015 --- Fall 2016
\BulletItem%
Java: Standard library, Collection / Stream interfaces, Swing
\BulletItem%
Data Structures: Linked Lists, Trees, Heaps, Graphs
\BulletItem%
Algorithms: Dijkstra's algorithm, tree / graph traversal
\BulletItem%
Course assistant for the past two semesters
\begin{detail}
\SubBulletItem%
Office hours several times each week
\SubBulletItem%
Grading assignments, exams, and final
\end{detail}
\GapNoBreak%

\href{https://www.cs.cornell.edu/courses/cs2800/2016sp/}
{\textbf{CS 2800}}: Discrete Structures
\hfill Spring 2016
\BulletItem%
Number Theory, Modular Arithmetic, RSA
\BulletItem%
Combinatorics, Probability, Graph Theory
\BulletItem%
DFAs, NFAs, Regex, Regular Languages
\BulletItem%
Proof Systems, Propositional and First Order Logic
\GapNoBreak%

%{\textbf{CS 4320}}: Database Systems
%\hfill Fall 2016
%\BulletItem%
%SQL

%%%% Sideprojects
% Various little things that I have coded in my spare time

\section%
{Projects}
{Projects}
{PDF:Projects}

\href{https://www.github.com/lehmacdj/ocaml-ed}
{\textbf{OCaml Ed}, github.com/lehmacdj/ocaml-ed}
\BulletItem%
An implementation of \textbf{ed}, the 1960s line editor, written using OCaml.
\BulletItem%
OCaml Core (Janestreet's alternate standard library for OCaml)
\GapNoBreak%

\href{https://www.github.com/lehmacdj/wiki_depth}
{\textbf{Wikipedia Depth}, github.com/lehmacdj/wiki\_depth}
\BulletItem%
Traverses Wikipedia in order to find the first cycle of links found.
\BulletItem%
Haskell, TagSoup (malformed HTML parser)
\GapNoBreak%

\href{https://www.github.com/lehmacdj/.dotfiles}
{\textbf{Dotfiles}, github.com/lehmacdj/.dotfiles}
\BulletItem%
My extensive, cross platform configuration for the command line environment
\BulletItem%
Completely automated installation processes; just clone repository and run
install script
\BulletItem%
\textbf{template} command to add simple boilerplate for new projects
new templates
\GapNoBreak%

\section%
{Research Experience}
{Research Experience}
{PDF:ResearchExperience}

\href{https://www.clemson.edu}
{\textbf{Clemson University}},
College of Science
\hfill
\DatestampYMD{2015}{06}{15} --- Present
\BulletItem%
Undergraduate Research Student, Physics and Astronomy
\begin{detail}
\SubBulletItem%
Project:
MedusaLoop: Protein Loop Modeling Server
\SubBulletItem%
Supervisors:
Dr.\ Feng Ding
\SubBulletItem%
Research areas:
Protein loops, web development, linux, shell scripting
\end{detail}

%%%%%%%%%%%%
%% SKILLS %%
%%%%%%%%%%%%

\section%
{Skills}
{Skills}
{PDF:Skills}

\BulletItem%
Java, OCaml, Haskell, Commandline tools, C, C++, SQL
\BulletItem%
Ability to learn new programming languages easily
\SubBulletItem%
Frequently learn syntax of other languages
\SubBulletItem%
Perl, Swift, Rust, Idris, Go, Scheme

%%%%%%%%%%%%%%%
%% LANGUAGES %%
%%%%%%%%%%%%%%%
%-
%- \section%
%- {Languages}
%- {Languages}
%- {PDF:Languages}
%-
%- \BulletItem%
%- English: Native language.
%-
%- \GapNoBreak%
%- \BulletItem%
%- German: Advanced (speaking, reading, writing).
%- \begin{detail}
%- \SubBulletItem%
%- Raised speaking German at home.
%- \end{detail}
%-
%- \GapNoBreak%
%- \BulletItem%
%- Spanish: Intermediate (reading, reading, writing).
%-
%- %%%%%%%%%%%%%%%
%- %% INTERESTS %%
%- %%%%%%%%%%%%%%%
%-
%- \section%
%- {Interests}
%- {Interests}
%- {PDF:Interests}
%-
%- Board games,
%- Outdoors,
%- Hearthstone,
%- Running.

\end{body}

% Note about when resume was compiled
%\begin{flushright}
%\UseNoteFont%
%[\textit{\CVNote}]%
%\hspace{2.0mm}\null%
%\end{flushright}

\label{LastPage}~%
\end{document}
