% !TEX TS-program = xelatex
% !TEX encoding = UTF-8

\documentclass[letterpaper,10pt,oneside]{simpleresume}

% CV Info (to be customized).
\newcommand{\CVAuthor}{Devin Lehmacher}
\newcommand{\CVCompany}{Jane Street}
\newcommand{\CVTitle}{\CVAuthor's Resume for \CVCompany}
\newcommand{\CVWebpage}{github.com/lehmacdj}

% PDF settings and properties.
\hypersetup{%
pdftitle={\CVTitle},
pdfauthor={\CVAuthor},
pdfsubject={\CVWebpage},
pdfcreator={XeLaTeX},
pdfproducer={},
pdfkeywords={},
pdfpagemode={},
bookmarks=true,
unicode=true,
bookmarksopen=true,
pdfstartview=FitH,
pdfpagelayout=OneColumn,
pdfpagemode=UseOutlines,
hidelinks,
breaklinks}

% Shorthand.
\newcommand{\CodeCommand}[1]{\mbox{\textbf{\textbackslash{#1}}}}

\begin{document}
\begin{minipage}[t][0pt]{\linewidth}
\pagestyle{empty}

\title{\CVAuthor}

\begin{subtitle}
\href{https://www.google.com/maps/place/112+Sage+Pl+Room-B09,+Ithaca,+NY+14850}
{112 Sage Place, Ithaca, NY 14850}
%\href{https://www.google.com/maps/place/232+Kings+Way,+Clemson,+SC+29631,+USA}
%{232 Kings Way, Clemson, SC 29631, USA}
\par
\href{mailto:djl329@cornell.edu}
{djl329@cornell.edu}
\,\SubBulletSymbol\,
+1\,(864)\,722--3014
\,\SubBulletSymbol\,
\href{https://\CVWebpage}
{\CVWebpage}
\end{subtitle}

\begin{body}

\section%
{Objective}
{Objective}
{PDF:Objective}
To obtain a summer internship at \CVCompany{} as a software engineer.

\section%
{Education}
{Education}
{PDF:Education}

\href{https://www.cornell.edu}
{\textbf{Cornell University}},
Ithaca, NY 14853
\hfill
\DatestampYMD{2015}{08}{25} --- Present
\BulletItem%
Expect to graduate \DatestampYM{2019}{05}
\BulletItem%
Cumulative GPA:\@ 3.45
\BulletItem%
Bachelors of Arts in
\href{https://www.cs.cornell.edu}{Computer Science}
\BulletItem%
Bachelors of Arts in
\href{https://www.biology.cornell.edu}{Biology}

\section%
{Classes}
{Classes}
{PDF:Classes}

\href{https://www.cs.cornell.edu/courses/cs3110/2016sp/}
{\textbf{Functional Programming and Data Structures}, CS 3110}
\hfill Spring 2016
\BulletItem%
OCaml: Standard library, Async, Functional programming techniques
% I have since learned OCaml Core this should be in the sideproject section
% where I have a part about OCaml Ed
% Using options and monads to make life easier
% Statelessness allows for much better concurrency and makes programs easier to
% reason about
\BulletItem%
Type theory, propositional logic, constructive real numbers, convex hull problem
% Value assurance, error handling using option types, monads, functors,
% assurance of invariants through the use of variant checking functions
% Proofs of propositional logic statements using evidence of the existence of a
% function
% Bishop book,
\BulletItem%
Data Structures: Splay trees, Monads, Modules, Functors
% Self balancing BST where most recently used stuff is near to top
% join, bind, return: important for usage of async
% modules as abstraction
% functors as ways to implement generic stuff and conform to protocols
\BulletItem%
Language evaluation: wrote an interpreter for a subset of OCaml
% Small step evaluation: follow order of operations from the leaves of the AST
% Big step evaluation: environment and such stuff; we used this to implement
% OCalf

\href{https://www.cs.cornell.edu/courses/cs3410/2016fa/}
{\textbf{Computer System Organization}, CS 3410}
\hfill Fall 2016
\BulletItem%
Digital design using Logisim, MIPS assembly, C
\BulletItem%
Processor design: built a fully pipelined MIPS processor
instructions
\BulletItem%
Memory management: implementation of malloc in C

\href{https://www.cs.cornell.edu/courses/cs2110/2015fa/}
{\textbf{Object Oriented Programming}, CS 2110}
\hfill Fall 2015
\BulletItem%
Java: Standard library, Collection / Stream interfaces, Swing
\BulletItem%
Data Structures: Linked Lists, Trees, Heaps, Graphs
\BulletItem%
Algorithms: Dijkstra's algorithm, tree / graph traversal

\href{https://www.cs.cornell.edu/courses/cs2800/2016sp/}
{\textbf{Discrete Structures}, CS 2800}
\hfill Spring 2016
\BulletItem%
Number Theory, Modular Arithmetic, RSA Encryption
\BulletItem%
Combinatorics, Probability, Graph Theory
\BulletItem%
DFAs, NFAs, Regex, Regular Languages
\BulletItem%
Proof Systems, Propositional and First Order Logic

%{\textbf{CS 4320}}: Database Systems
%\hfill Fall 2016
%\BulletItem%
%SQL

\section%
{Projects}
{Projects}
{PDF:Projects}

\href{https://www.github.com/lehmacdj/ocaml-ed}
{\textbf{OCaml Ed}, github.com/lehmacdj/ocaml-ed}
\BulletItem%
Implementation of \textbf{ed}, the 1960s line editor, written using OCaml
\BulletItem%
\href{https://www.github.com/janestreet/core}
{OCaml Core},
\href{https://www.github.com/janestreet/re2}
{Re2}

\href{https://www.github.com/lehmacdj/wiki_depth}
{\textbf{Wikipedia Depth}, github.com/lehmacdj/wiki\_depth}
\BulletItem%
Traverses Wikipedia in order to find the first cycle of links found
\BulletItem%
\href{https://www.haskell.org}
{Haskell},
\href{https://hackage.haskell.org/package/tagsoup}
{TagSoup (malformed HTML parser)}

\href{https://www.github.com/lehmacdj/.dotfiles}
{\textbf{Dotfiles}, github.com/lehmacdj/.dotfiles}
\BulletItem%
Extensive, cross platform configuration for the command line environment
\BulletItem%
Completely automated installation processes; just clone repository and run
install script
\BulletItem%
\textbf{template} command to add simple boilerplate for new projects
new templates

\section%
{Work Experience}
{Work Experience}
{PDF:WorkExperience}

\textbf{Course Assistant}, CS 2110 at Cornell University
\hfill
Spring 2016 --- Present
\BulletItem%
Help explain concepts to students
\BulletItem%
Assist students with assignments
\BulletItem%
Grade assignments, exams, and finals

\textbf{Research Assistant} at Clemson University
\hfill
\DatestampYMD{2015}{06}{15} --- \DatestampYMD{2016}{08}{15}
\BulletItem%
Project:
MedusaLoop: Protein Loop Modeling Server
\BulletItem%
Supervisor:
Dr.\ Feng Ding
\BulletItem%
Research areas:
Protein loop modeling

\section%
{Skills}
{Skills}
{PDF:Skills}

\textbf{Programming Languages}
\BulletItem%
Java, OCaml, Haskell, Commandline tools, C, C++, SQL
\BulletItem%
Ability to learn new programming languages easily
\begin{detail}
\SubBulletItem%
Frequently learn syntax of new languages
\SubBulletItem%
Perl, Swift, Rust, Idris, Go, Scheme
\end{detail}
\textbf{Languages}
\BulletItem%
English (fluent), German (fluent), Spanish (intermediate)
\end{body}

\end{minipage}
\end{document}
