% !TEX TS-program = xelatex
% !TEX encoding = UTF-8

\documentclass[letterpaper,10pt,oneside]{simpleresume}

% CV Info (to be customized).
\newcommand{\CVAuthor}{Devin Lehmacher}
\newcommand{\CVTitle}{\CVAuthor{}'s Resume}
\newcommand{\CVWebpage}{github.com/lehmacdj}

% PDF settings and properties.
\hypersetup{%
pdftitle={\CVTitle},
pdfauthor={\CVAuthor},
pdfsubject={\CVWebpage},
pdfcreator={XeLaTeX},
pdfproducer={},
pdfkeywords={},
pdfpagemode={},
bookmarks=true,
unicode=true,
bookmarksopen=true,
pdfstartview=FitH,
pdfpagelayout=OneColumn,
pdfpagemode=UseOutlines,
hidelinks,
breaklinks}

% Shorthand.
\newcommand{\CodeCommand}[1]{\mbox{\textbf{\textbackslash{#1}}}}

\begin{document}
\begin{minipage}[t][0pt]{\linewidth}
\pagestyle{empty}

\title{\CVAuthor}

\begin{subtitle}
\href{https://www.google.com/maps/place/112+Sage+Pl+Room-209,+Ithaca,+NY+14850}
{112 Sage Place Room-209, Ithaca, NY 14850}
\par
\href{mailto:djl329@cornell.edu}
{djl329@cornell.edu}
\,\SubBulletSymbol\,
+1\,(864)\,722--3014
\,\SubBulletSymbol\,
\href{https://\CVWebpage}
{\CVWebpage}
\end{subtitle}

\begin{body}

\section%
{Education}
{Education}
{PDF:Education}

\href{https://www.cornell.edu}
{\textbf{Cornell University}}, Ithaca, NY
\hfill
\DatestampYMD{2015}{08}{25} --- Present
\BulletItem%
Graduating in \DatestampYM{2019}{05}
\BulletItem%
Bachelors of Arts in
\href{https://www.cs.cornell.edu}{Computer Science}
\BulletItem%
Cumulative GPA:\@ 3.52, Major GPA:\@ 3.95

\section%
{Classes}
{Classes}
{PDF:Classes}

\Class{Introduction to Compilers \& Practicum}{CS 4120 \& CS 4121}
\Class{Certified Software Systems}{CS 6115}
\Class{Constructive Type Theory}{CS 6180}
\Class{Intro to Analysis of Algorithms}{CS 4820}
\Class{Intro to Theory of Computing}{CS 4810}
\Class{Advanced Programming Languages}{CS 6110}
\Class{Operating Systems \& Practicum}{CS 4410 \& CS 4411}
\Class{Database Systems}{CS 4320}
\Class{Computer System Organization}{CS 3410}
\Class{Functional Programming and Data Structures}{CS 3110}
\Class{Discrete Structures}{CS 2800}
\Class{Object Oriented Programming and Data Structures}{CS 2110}

\section%
{Work Experience}
{Work Experience}
{PDF:WorkExperience}

\textbf{Teaching Assistant}, CS 2110 at Cornell University
\hfill
\DatestampYM{2016}{2} --- Present
\BulletItem%
Teach a section with about 25 students each week
\BulletItem%
Hold weekly office hours to help students understand the course material
\BulletItem%
Help test, create, and plan future assignments
\BulletItem%
Grade assignments and exams, giving students helpful feedback

\textbf{Intern} at Itron Inc.\ in Seneca, SC
\hfill
\DatestampYMD{2017}{06}{23} --- \DatestampYMD{2016}{08}{18}
\BulletItem%
Created a dashboard to visualize available space for testing meters
\BulletItem%
Utilized Transact-SQL to collect data for the dashboard
\BulletItem%
Built and deployed a report to Sharepoint using Microsoft Reporting Services

\textbf{Research Assistant} at Clemson University
\hfill
\DatestampYMD{2015}{06}{15} --- \DatestampYMD{2016}{08}{15}
\BulletItem%
Tested the performance of MedusaLoop, a program that models protein loops
\BulletItem%
Analyzed test results to visualize performance
\BulletItem%
Wrote a daemon to dispatch jobs from a database to a server instance
\BulletItem%
Wrote back end code that interacted with a database to fetch and write new jobs

\section%
{Projects}
{Projects}
{PDF:Projects}

\textbf{PortOS}, CS 4411
\BulletItem%
Implemented multithreading with preemption, and TCP and UDP analogs
\BulletItem%
Learned how to navigate and write a large (10,000 lines) C code base
\BulletItem%
Wrote safe, concurrent, robust C code
\GapNoBreak%

\textbf{OCalf Interpreter}, CS 3110
\BulletItem%
Built an interpreter for a small subset of OCaml
\BulletItem%
Learned how to evaluate an AST for a functional language using small step
semantics
\BulletItem%
Implemented Hindley-Milner type inference algorithm to type check OCalf programs
\GapNoBreak%

\textbf{Scheme Interpreter}, github.com/lehmacdj/haskell\_scheme
\BulletItem%
Built an interpreter for a subset of Scheme
\BulletItem%
Learned how to implement the semantics for dynamically typed programming
languages
\BulletItem%
Learned how to build a parser using Parsec
\GapNoBreak%

\textbf{Heaplib}, CS 3410
\BulletItem%
Implemented and tested malloc, free, and resize in C
\BulletItem%
Learned how to use raw pointers and the trade-offs involved with building
an allocator
\BulletItem%
Wrote a large number of tests to ensure that pointer arithmetic was correct
\GapNoBreak%

% \textbf{MIPS Processor}, CS 3410
% \BulletItem%
% Designed a MIPS processor in Logisim and tested it with programs written in
% assembly
% \BulletItem%
% Learned how to decode binary MIPS instructions into control signals
% \GapNoBreak%

% \textbf{OCaml Ed}, github.com/lehmacdj/ocaml-ed
% \BulletItem%
% Implementation of ed, the 1960s line editor, written using OCaml
% \BulletItem%
% Wrote clear error handling code that cleanly passes errors up to the top level
% \GapNoBreak%

% \href{https://www.github.com/lehmacdj/wiki_depth}
% {\textbf{Wikipedia Depth}, github.com/lehmacdj/wiki\_depth}
% \BulletItem%
% Implemented a program that traverses Wikipedia to find the first encountered
% cycle
% \BulletItem%
% Learned how to access and parse websites using Haskell

% \textbf{Life Simulator}, github.com/lehmacdj/simulation
% \BulletItem%
% Implemented the Game of Life and multicolor variants using Rust
% \BulletItem%
% Learned how to write memory safe code using Rust and generate png images

\section%
{Skills}
{Skills}
{PDF:Skills}

Fluent: Java, Haskell, git, Vim, C, OCaml, Rust
\GapNoBreak%
Familiar: SQL, shell scripting, C++, Python, Perl

\end{body}
\end{minipage}
\end{document}
